\documentclass{article}

\usepackage{amsmath}
\usepackage[margin=0.5in]{geometry}
\usepackage[hidelinks]{hyperref}
\usepackage{titlesec}
\usepackage[T1]{fontenc}

\geometry{a4paper, margin=0.5in}
\urlstyle{same}

\newcommand{\cdelim}{\;\textbar\;}
\newcommand{\newrole}[4]{
    {\normalfont\bfseries #1\hfill#3}
    \newline
    \textit{#2}\hfill\textit{#4}
}

\newenvironment{bulletpoints}{\vspace*{-0.2em}\begin{itemize}\setlength\itemsep{-0.3em}}{\end{itemize}}

\titleformat{\section}{\large\bfseries}{\thesection}{0em}{\vspace*{-0.3em}}[\titlerule\vspace*{-0.2em}]

\setlength\parindent{0pt}

\pagestyle{empty}



\begin{document}

\begin{center}
    {\Huge\bfseries Justin Bax}\\\vspace*{2pt}

    Montréal, Quebec\cdelim 438.763.6066\cdelim justin.bax@icloud.com\cdelim\href{https://github.com/justinbax}{github.com/justinbax}\cdelim\href{https://linkedin.com/in/justin-bax}{linkedin.com/in/justin-bax}\\
\end{center}

\section*{Éducation}

\newrole{Vanier College}{Diplôme d'études collégiales, Computer Science \& Mathematics}{Montréal, Quebec}{Classe de 2025}
\begin{bulletpoints}
    \item 97\% de moyenne académique; bourse d'entrée pour excellente performance académique; \textit{Dean's Honour Roll} 
\end{bulletpoints}

\newrole{Collège Jean-Eudes}{Diplôme d'études secondaires}{Montréal, Quebec}{Classe de 2023}
\begin{bulletpoints}
    \item 94\% de moyenne académique; bourse de 500\$ pour réussite remarquable en musique et dans les études 
\end{bulletpoints}


\section*{Compétences techniques}

{\bfseries Languages de programmation}: C/C++, Python, Java, JavaScript, HTML/CSS, Rust, 6502 Assembly
\newline
{\bfseries Tech/Outils}: SQL, MongoDB, Git, REST API, Google Cloud API, Linux, Command line, OpenGL, Flask, NumPy


\section*{Projets}

\newrole{Spinich}{Application Web facilitant la recherche d'emploi par cold emails avec IA}{Montréal, Quebec}{Janvier 2024}
\begin{bulletpoints}
    \item Développement du backend \& REST API d’une application Web automatisant l’envoi de cold emails personnalisés
    \item Surveillance constante des courriels de l’utilisateur; analyse par IA des réponses obtenues pour maximiser l’efficacité
    \item 3$^e$\hspace*{-0.1em} place et 2 prix commandités à BrébeufHx. Approché par la startup ayant créé le défi pour le potentiel innovant
\end{bulletpoints}

\newrole{SingularIO}{Soumission gagnante pour le McGill Physics Hackathon 2023}{Montréal, Quebec}{Novembre 2023}
\begin{bulletpoints}
    \item Développement d’une simulation n-body interactive et d’une visualisation de la distorsion de l’espace-temps
    \item Choisi parmi 140 participants pour gagner la première place et le People’s Choice award. Utilise PyGame et NumPy
\end{bulletpoints}

\newrole{NESRev}{Émulateur de la NES au cycle précis et moteur de rendu}{Montréal, Quebec}{Août 2021 --- Mars 2022}
\begin{bulletpoints}
    \item Développement en solo d’un émulateur de la Nintendo réalisant une précision au cycle près. Inclut un mode d’exécution pas-à-pas, de l’information de débogage, un script de création de fichier ROM à partir de code source et la génération d’audio
    \item Programmé en C. Fonctionne en utilisant mon propre rendering engine OpenGL
\end{bulletpoints}

\section*{Expérience professionnelle}

\newrole{Julie Plante Computer Science Laboratory}{Stage de recherche en IA}{Montréal, Quebec}{Septembre 2024 --- Mai 2025}
\begin{bulletpoints}
    \item Prévu de compléter un stage de recherche en IA/LLM de 32 semaines durant l’année académique 2024-2025
    \item Sélectionné parmi tous les étudiants de science à Vanier pour recevoir une bourse de recherche collégiale du FRQNT
\end{bulletpoints}

\newrole{Chez Cora}{Serveur}{Montréal, Quebec}{Mai 2021 --- Octobre 2023}
\begin{bulletpoints}
    \item Tâches: gérer le service à la clientèle avec une grande efficacité et un service amical durant les heures d’affluence
    \item A reçu une promotion de commis débarrasseur à aide serveur après un an, puis une autre d’aide-serveur à serveur
\end{bulletpoints}


\section*{Expérience de leadership}
\newrole{FLOSS (Open-Source) Club}{Organisateur}{Montréal, Quebec}{Septembre 2023 --- Présent}
\begin{bulletpoints}
    \item Organisateur d’un évènement de style anti-conférence d’un jour complet avec un thème \textit{libre}/\textit{open-source}
    \item A tenu un atelier pour plus de 20 participants à propos de divers usages du Raspberry Pi reliés au \textit{networking}
    \item A organisé un marathon d’inventaire informatique, me permettant d’exploiter mes compétences en diagnostic, résolution de problèmes, \textit{command line}, Linux et Windows
    \item Sélectionné pour contribuer à une étude sur la collection et l’analyse de données liées au processus d’installation de Debian. Prévoit la production d’un rapport officiel pour améliorer l’accessibilité aux logiciels et aux projets \textit{open-source}
\end{bulletpoints}


\section*{Informations additionnelles}
{\bfseries Activitiés}: Tutorat, leader d’ensemble jazz, compétitions de trombone classique, concours de mathématiques
\newline
{\bfseries Intérêts}: Badminton, échecs, physique quantique, mandarin, littérature de non-fiction, théorie des jeux, jiu-jitsu

\end{document}